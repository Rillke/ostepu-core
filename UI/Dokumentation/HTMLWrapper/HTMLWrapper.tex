\section{Die Klasse HTMLWrapper}
\label{sub:Die-Klasse-HTMLWrapper}
Damit nicht jedes Template sein eigenes HTML Grundgerüst bieten muss kann die
Klasse HTMLWrapper mehrere Elemente die auf einer Seite angezeigt werden sollen
in ein HTML Gerüst einbetten. Außerdem bietet sie die Möglichkeit Elemente in
Formulare einzubetten.

Es ist möglich Scripte und Stylesheets die eine Seite
beeinhalten soll in einer Konfigutrationsdatei einzutragen und diese automatisch
einbinden lassen. Eine solche Konfigurationsdatei ist in folgendem Listing zu
sehen.

\mylisting[language=json]{Listings/HTMLWrapper-config.json}

Die Benutzung der Klasse ist wie folgt:

\mylisting[language=myhtml]{Listings/HTMLWrapper-template1.html}
\mylisting[language=myhtml]{Listings/HTMLWrapper-template2.html}
\mylisting[language=php]{Listings/HTMLWrapper-usage.php}

Dieser Code erzeugt folgende Ausgabe:

\mylisting[language=myhtml]{Listings/HTMLWrapper-result.html}
