\section{Templates}
\label{sub:Templates}
Die Seiten der Übungsplattform sind aus (eventuell mehreren) Templates
aufgebaut. Ein Template ist ein php-Skript, das abhängig von den durch das
Template dargestellten Daten vordefinierte Variablen kennt. Daten die an ein
Template übergeben werden sind assoziative Arrays, die Schlüssel des Arrays sind
im Template als Variablen verfügbar. Die folgenden listings zeigen ein
JSON-codiertes assoziatives Array und ein mögliches Template zur Darstellung der
Daten in dem Array.

\mylisting[language=json]{Listings/template-json.json}
\mylisting[language=myphp]{Listings/template-html.html}

Mit dem Code in folgendem Listing könnte schon eine Seite dargestellt
werden

\mylisting[language=myphp]{Listings/template-usage.php}


