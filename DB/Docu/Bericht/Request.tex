\section{HTTP Requests erstellen}
Für HTTP-Requests nutzen wir cURL, dazu wurden Hilfsklassen entwickelt, welche den Zugriff darauf erlauben. Alles notwendige dazu finden wir in den Dateien \blau{Assistants/Request.php} und \blau{Assistants/Request/CreateRequest.php}.

\subsection{Anfragen für Anfänger}
\begin{minipage}{\textwidth}
\begin{lstlisting}
    public static function custom($method, 
                                  $target, 
                                  $header,  
                                  $content)
    {
        ...
    }
\end{lstlisting}
\end{minipage}

Dabei werden GET,PUT und POST unterstützt, zudem gibt es auch die Möglichkeit, eigene Methoden anzugeben.
\blau{Assistants/Request.php}

\begin{minipage}{\textwidth}
Beispiel:
\begin{lstlisting}
$ch = Request::custom(
                  'GET', // Methode
                  'http://.../DBCourse/course', // Anfrageziel
                  array(), // Header
                  ''); // Content
echo $ch['status'];
echo $ch['content']; 
                     
// Ausgabe: (nur beispielhaft)
//          200
//          []         
\end{lstlisting}
\end{minipage}
Auf den Header greifen Sie über \blau{\$ch['headers']} zu.

\subsection{Anfragen mit Verknüpfungen}
\begin{minipage}{\textwidth}
\begin{lstlisting}
    public static function routeRequest($method, 
                                        $resourceUri, 
                                        $header,  
                                        $content,
                                        $linkedComponents,
                                        $prefix,
                                        $linkName=NULL)
    {
        ...
    }
\end{lstlisting}
\end{minipage}

Wenn wir eine Liste von $Link$ Objekten zur Verfügung haben, so können wir die \blau{Request::routeRequest()} Funktion nutzen, um das Ziel unserer Anfrage aus mehreren Verknüpfungen suchen zu lassen.
\blau{Assistants/Request.php}

\begin{minipage}{\textwidth}
Beispiel:
\begin{lstlisting}
$ch = Request::routeRequest("GET", // Methode
                            '/course', // Anfrage-URI
                            array(), // Header
                            '', // Content
                            $links); // Linkliste
echo $ch['status'];
echo $ch['content']; 
                     
// Ausgabe: (nur beispielhaft)
//          200
//          []
\end{lstlisting}
\end{minipage}

\subsection{Anfragen mit SQL Dateien}
\begin{minipage}{\textwidth}
\begin{lstlisting}
    public static function getRoutedSqlFile($querys, 
                                            $sqlFile, 
                                            $vars,
                                            $checkSession=true)
    {
    	...
    }
\end{lstlisting}
\end{minipage}
Wenn wir eine Liste von $Link$ Objekten zur Verfügung haben und den Inhalt einer SQL Datei mit der Datenbank über die $DBQuery$ Komponente ausführen wollen, so können wir die \blau{DBRequest::getRroutedSqlFile()} Funktion nutzen.
Dabei können zusätzlich Variablen als Array übergeben werden, die dann innerhalb der SQL Datei global zur Verfügung stehen, da der Inhalt der SQL Datei mit PHP interpretiert wird über \blau{eval()}.

Es kann optional angegeben werden ob in der $Query$ Struktur, die zur $DBQuery$ Komponente geschickt wird, das Flag zur Prüfung der Session abgeschaltet werden soll.
\blau{Assistants/DBRequest.php}

\begin{minipage}{\textwidth}
Beispiel:
\begin{lstlisting}
$ch = DBRequest::getRoutedSqlFile($links, // Linkliste
                    'Beispiel.sql', // SQL Datei
                    array('userid' => $userid, // Platzhalter
                          'courseid' => $courseid),
                    false); // checksession ist hier aus
echo $ch['status'];
echo $ch['content']; 
                     
// Ausgabe: (nur beispielhaft)
//          200
//          []
\end{lstlisting}
\end{minipage}